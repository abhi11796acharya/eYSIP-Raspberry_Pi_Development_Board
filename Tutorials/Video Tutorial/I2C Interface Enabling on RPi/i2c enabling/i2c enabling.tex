\documentclass[10pt,red]{beamer} 
% change the alerted colour to blue
\setbeamercolor{alerted text}{fg=blue}

\usetheme{berlin}
% theme split
\usepackage{beamerthemesplit}

\usepackage{booktabs,array,}
\usepackage{listings}
\usepackage{hyperref}
\usepackage{verbatim,moreverb}
\usepackage{tikz}

\usepackage{color}

\definecolor{dkgreen}{rgb}{0,0.6,0}
\definecolor{gray}{rgb}{0.5,0.5,0.5}
\definecolor{mauve}{rgb}{0.58,0,0.82}

\lstset{frame=tb,
  language = Java,
  aboveskip=3mm,
  belowskip=3mm,
  showstringspaces=true,
  columns=flexible,
  basicstyle={\small\ttfamily},
  numbers=none,
  numberstyle=\tiny\color{gray},
  keywordstyle=\color{blue},
  commentstyle=\color{dkgreen},
  stringstyle=\color{mauve},
  breaklines=true,
  breakatwhitespace=true
  tabsize=4
}
% theme shadow
\usepackage{beamerthemeshadow}

% For including figures
\usepackage{graphicx}

% logo
\logo{\includegraphics[height=1cm]{iitblogo.pdf}}


% sf family, bold font
\sffamily \bfseries
% Beginning of title page
\title
% content inside [] appears at bottom of all page. content inside {} appears on first page as title. double backslash means line change 
[
	Raspberry Pi Hardware Development	% bottom
	\hspace{0.5cm}
	\insertframenumber/\inserttotalframenumber
]
{
	I2C enabling on Raspberry Pi
}

\author
[
	www.e-yantra.org
]
{
	e-Yantra Team \\
  Embedded Real-Time Systems Lab\\
  Indian Institute of Technology-Bombay \\
}
\date
{
IIT Bombay \\ {\today}
}
 
 
\begin{document} 

% Slide-1: Title Page
\begin{frame}
	\titlepage
\end{frame} 
\section{About I2C}
\begin{frame}
	\frametitle{What is I2C?}  \pause
	\begin{enumerate}[$\checkmark$]
		\item<+-|alert@+> I2C stands for Inter Integrated Circuits.
		\item<+-|alert@+> It is a protocol for communication between two devices.
		\item<+-|alert@+> I2C is a multi-master protocol that uses 2 signal lines.
		\item<+-|alert@+> The two I2C signals are called ‘serial data’ (SDA) and ‘serial clock’ (SCL).
		\item<+-|alert@+> Any number of master and slaves can be connected to these two lines.
		\item<+-|alert@+> Each device connected to the bus will get a unique address.
		\item<+-|alert@+> The IC that initiates a data transfer on the bus is considered the Bus Master.
		
	\end{enumerate}
\end{frame}

\begin{frame}
	\hskip4cm
	\textbf{\LARGE Thank You!} \\[20pt]
	\hskip3cm
	\scriptsize Post your queries on: 
	\hyperref[www.e-yantra.org]{\color{blue} http://qa.e-yantra.org/ \color{black}} 
\end{frame}
\end{document}